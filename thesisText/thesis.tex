%% History:
% Pavel Tvrdik (26.12.2004)
%  + initial version for PhD Report
%
% Daniel Sykora (27.01.2005)
%
% Michal Valenta (3.12.2008)
% rada zmen ve formatovani (diky M. Duškovi, J. Holubovi a J. Žďárkovi)
% sjednoceni zdrojoveho kodu pro anglickou, ceskou, bakalarskou a diplomovou praci

% One-page layout: (proof-)reading on display
%%%% \documentclass[11pt,oneside,a4paper]{book}
% Two-page layout: final printing
\documentclass[11pt,twoside,a4paper]{book}   
%=-=-=-=-=-=-=-=-=-=-=-=--=%
% The user of this template may find useful to have an alternative to these 
% officially suggested packages:
\usepackage[czech, english]{babel}
\usepackage[T1]{fontenc} % pouzije EC fonty 
% pripadne pisete-li cesky, pak lze zkusit take:
% \usepackage[OT1]{fontenc} 
\usepackage[utf8]{inputenc}
%=-=-=-=-=-=-=-=-=-=-=-=--=%
% In case of problems with PDF fonts, one may try to uncomment this line:
%\usepackage{lmodern}
%=-=-=-=-=-=-=-=-=-=-=-=--=%
%=-=-=-=-=-=-=-=-=-=-=-=--=%
% Depending on your particular TeX distribution and version of conversion tools 
% (dvips/dvipdf/ps2pdf), some (advanced | desperate) users may prefer to use 
% different settings.
% Please uncomment the following style and use your CSLaTeX (cslatex/pdfcslatex) 
% to process your work. Note however, this file is in UTF-8 and a conversion to 
% your native encoding may be required. Some settings below depend on babel 
% macros and should also be modified. See \selectlanguage \iflanguage.
%\usepackage{czech}  %%%%%\usepackage[T1]{czech} %%%%[IL2] [T1] [OT1]
%=-=-=-=-=-=-=-=-=-=-=-=--=%

%%%%%%%%%%%%%%%%%%%%%%%%%%%%%%%%%%%%%%%
% Styles required in your work follow %
%%%%%%%%%%%%%%%%%%%%%%%%%%%%%%%%%%%%%%%
\usepackage{graphicx}
%\usepackage{indentfirst} %1. odstavec jako v cestine.

\usepackage{k336_thesis_macros} % specialni makra pro formatovani DP a BP
 % muzete si vytvorit i sva vlastni v souboru k336_thesis_macros.sty
 % najdete  radu jednoduchych definic, ktere zde ani nejsou pouzity
 % napriklad: 
 % \newcommand{\bfig}{\begin{figure}\begin{center}}
 % \newcommand{\efig}{\end{center}\end{figure}}
 % umoznuje pouzit prikaz \bfig namisto \begin{figure}\begin{center} atd.


%%%%%%%%%%%%%%%%%%%%%%%%%%%%%%%%%%%%%
% Zvolte jednu z moznosti 
% Choose one of the following options
%%%%%%%%%%%%%%%%%%%%%%%%%%%%%%%%%%%%%
%\newcommand\TypeOfWork{Diplomová práce} \typeout{Diplomova prace}
 \newcommand\TypeOfWork{Master's Thesis}   \typeout{Master's Thesis} 
% \newcommand\TypeOfWork{Bakalářská práce}  \typeout{Bakalarska prace}
% \newcommand\TypeOfWork{Bachelor's Project}  \typeout{Bachelor's Project}


%%%%%%%%%%%%%%%%%%%%%%%%%%%%%%%%%%%%%
% Zvolte jednu z moznosti 
% Choose one of the following options
%%%%%%%%%%%%%%%%%%%%%%%%%%%%%%%%%%%%%
% nabidky jsou z: http://www.fel.cvut.cz/cz/education/bk/prehled.html

%\newcommand\StudProgram{Elektrotechnika a informatika, dobíhající, Bakalářský}
%\newcommand\StudProgram{Elektrotechnika a informatika, dobíhající, Magisterský}
% \newcommand\StudProgram{Elektrotechnika a informatika, strukturovaný, Bakalářský}
% \newcommand\StudProgram{Elektrotechnika a informatika, strukturovaný,
% Navazující magisterský}
% \newcommand\StudProgram{Softwarové technologie a management, Bakalářský}
% English study:
% \newcommand\StudProgram{Electrical Engineering and Information Technology}  % bachelor programe
 \newcommand\StudProgram{Electrical Engineering and Information Technology}  %master program


%%%%%%%%%%%%%%%%%%%%%%%%%%%%%%%%%%%%%
% Zvolte jednu z moznosti 
% Choose one of the following options
%%%%%%%%%%%%%%%%%%%%%%%%%%%%%%%%%%%%%
% nabidky jsou z: http://www.fel.cvut.cz/cz/education/bk/prehled.html

%\newcommand\StudBranch{Výpočetní technika}   % pro program EaI bak. (dobihajici i strukt.)
%\newcommand\StudBranch{Výpočetní technika}   % pro prgoram EaI mag. (dobihajici
%i strukt.)
%\newcommand\StudBranch{Softwarové inženýrství}            %pro STM
%\newcommand\StudBranch{Web a multimedia}                  % pro STM
%\newcommand\StudBranch{Computer Engineering}              % bachelor programe
\newcommand\StudBranch{Computer Science and Engineering}  % master programe


%%%%%%%%%%%%%%%%%%%%%%%%%%%%%%%%%%%%%%%%%%%%
% Vyplnte nazev prace, autora a vedouciho
% Set up Work Title, Author and Supervisor
%%%%%%%%%%%%%%%%%%%%%%%%%%%%%%%%%%%%%%%%%%%%

\newcommand\WorkTitle{Framework Squander Usage}
\newcommand\FirstandFamilyName{Bc. Martin Kožený}
\newcommand\Supervisor{Ing. Jiří Daněček}


% Pouzijete-li pdflatex, tak je prijemne, kdyz bude mit vase prace
% funkcni odkazy i v pdf formatu
\usepackage[
pdftitle={\WorkTitle},
pdfauthor={\FirstandFamilyName},
bookmarks=true,
colorlinks=true,
breaklinks=true,
urlcolor=black,
citecolor=black,
linkcolor=black,
unicode=true,
]
{hyperref}



% Extension posted by Petr Dlouhy in order for better sources reference (\cite{} command) especially in Czech.
% April 2010
% See comment over \thebibliography command for details.

\usepackage[square, numbers]{natbib}             % sazba pouzite literatury
%\usepackage{url}
%\DeclareUrlCommand\url{\def\UrlLeft{<}\def\UrlRight{>}\urlstyle{tt}}  %rm/sf/tt
%\renewcommand{\emph}[1]{\textsl{#1}}    % melo by byt kurziva nebo sklonene,
\let\oldUrl\url
\renewcommand\url[1]{<\texttt{\oldUrl{#1}}>}




\begin{document}

%%%%%%%%%%%%%%%%%%%%%%%%%%%%%%%%%%%%%
% Zvolte jednu z moznosti 
% Choose one of the following options
%%%%%%%%%%%%%%%%%%%%%%%%%%%%%%%%%%%%%
%\selectlanguage{czech}
\selectlanguage{english} 

% prikaz \typeout vypise vyse uvedena nastaveni v prikazovem okne
% pro pohodlne ladeni prace


\iflanguage{czech}{
	 \typeout{************************************************}
	 \typeout{Zvoleny jazyk: cestina}
	 \typeout{Typ prace: \TypeOfWork}
	 \typeout{Studijni program: \StudProgram}
	 \typeout{Obor: \StudBranch}
	 \typeout{Jmeno: \FirstandFamilyName}
	 \typeout{Nazev prace: \WorkTitle}
	 \typeout{Vedouci prace: \Supervisor}
	 \typeout{***************************************************}
	 \newcommand\Department{Katedra počítačů}
	 \newcommand\Faculty{Fakulta elektrotechnická}
	 \newcommand\University{České vysoké učení technické v Praze}
	 \newcommand\labelSupervisor{Vedoucí práce}
	 \newcommand\labelStudProgram{Studijní program}
	 \newcommand\labelStudBranch{Obor}
}{
	 \typeout{************************************************}
	 \typeout{Language: english}
	 \typeout{Type of Work: \TypeOfWork}
	 \typeout{Study Program: \StudProgram}
	 \typeout{Study Branch: \StudBranch}
	 \typeout{Author: \FirstandFamilyName}
	 \typeout{Title: \WorkTitle}
	 \typeout{Supervisor: \Supervisor}
	 \typeout{***************************************************}
	 \newcommand\Department{Department of Computer Science and Engineering}
	 \newcommand\Faculty{Faculty of Electrical Engineering}
	 \newcommand\University{Czech Technical University in Prague}
	 \newcommand\labelSupervisor{Supervisor}
	 \newcommand\labelStudProgram{Study Programme} 
	 \newcommand\labelStudBranch{Field of Study}
}




%%%%%%%%%%%%%%%%%%%%%%%%%%    Poznamky ke kompletaci prace
% Nasledujici pasaz uzavrenou v {} ve sve praci samozrejme 
% zakomentujte nebo odstrante. 
% Ve vysledne svazane praci bude nahrazena skutecnym 
% oficialnim zadanim vasi prace.
{
\pagenumbering{roman} \cleardoublepage \thispagestyle{empty}
\chapter*{Na tomto místě bude oficiální zadání vaší práce}
\begin{itemize}
\item Toto zadání je podepsané děkanem a vedoucím katedry,
\item musíte si ho vyzvednout na studiijním oddělení Katedry počítačů na Karlově náměstí,
\item v jedné odevzdané práci bude originál tohoto zadání (originál zůstává po obhajobě na katedře),
\item ve druhé bude na stejném místě neověřená kopie tohoto dokumentu (tato se vám vrátí po obhajobě).
\end{itemize}
\newpage
}

%%%%%%%%%%%%%%%%%%%%%%%%%%    Titulni stranka / Title page 

\coverpagestarts

%%%%%%%%%%%%%%%%%%%%%%%%%%%    Podekovani / Acknowledgements 

\acknowledgements
\noindent
At this point I would like to thank supervisor of my thesis Mr Ing. Jiří Daněček
and author of Squander framework Mr Aleksandar Milicevics for valuable comments
and advices to this work.


%%%%%%%%%%%%%%%%%%%%%%%%%%%   Prohlaseni / Declaration 

\declaration{In~Prague, 30.\,9.\,2011}
%\declaration{In Kořenovice nad Bečvárkou on May 15, 2008}


%%%%%%%%%%%%%%%%%%%%%%%%%%%%    Abstract 
 
\abstractpage

The aim of this work is to describe and study framework Squander developed on
Massachusetts Institute of Technology by Mr. Aleksandar Milicevic. This
framework brings into language Java another way of programming, which can
improve effectivness of implementing and computation performance of the program.

Work shows how was this framework used for implementing set of algorithms,
especially NP-complete graph algorithms, and compares that implementation with
common imperative way of programming.
% Prace v cestine musi krome abstraktu v anglictine obsahovat i
% abstrakt v cestine.
\vglue60mm

\noindent{\Huge \textbf{Abstrakt}}
\vskip 2.75\baselineskip

\noindent
Abstrakt práce by měl velmi stručně vystihovat její podstatu. Tedy čím se práce zabývá a co je jejím výsledkem/přínosem.

\noindent
Očekávají se cca 1 -- 2 odstavce, maximálně půl stránky.

%%%%%%%%%%%%%%%%%%%%%%%%%%%%%%%%  Obsah / Table of Contents 

\tableofcontents


%%%%%%%%%%%%%%%%%%%%%%%%%%%%%%%  Seznam obrazku / List of Figures 

\listoffigures


%%%%%%%%%%%%%%%%%%%%%%%%%%%%%%%  Seznam tabulek / List of Tables

\listoftables


%**************************************************************

\mainbodystarts
% horizontalní mezera mezi dvema odstavci
%\parskip=5pt
%11.12.2008 parskip + tolerance
\normalfont
\parskip=0.2\baselineskip plus 0.2\baselineskip minus 0.1\baselineskip

% Odsazeni prvniho radku odstavce resi class book (neaplikuje se na prvni 
% odstavce kapitol, sekci, podsekci atd.) Viz usepackage{indentfirst}.
% Chcete-li selektivne zamezit odsazeni 1. radku nektereho odstavce,
% pouzijte prikaz \noindent.

%**************************************************************

% Pro snadnejsi praci s vetsimi texty je rozumne tyto rozdelit
% do samostatnych souboru nejlepe dle kapitol a tyto potom vkladat
% pomoci prikazu \include{jmeno_souboru.tex} nebo \include{jmeno_souboru}.
% Napr.:
% \include{1_uvod}
% \include{2_teorie}
% atd...

%*****************************************************************************
\chapter{Introduction}
As was said in abstract, the main purpose of this thesis is to describe and use
framework Squander developed at Massachusetts Institute of Technology (MIT) by Mr
Aleksandar Milicevics. This framework brings to Java declarative
constructs, which are useful for implementing programs that involve computations
that are relatively easy to specify but hard to solve algorithmically. In such
cases is better to use declarative constraints to naturally express the core of
the computation, whereas imperative code is natural choice to read input
parameters and setting up data structures for the computation. This is
big advantage of this framework, that programmer can smoothly switch between
declarative logical and imperative programming.

By ability of mixing imperative and declarative code can programmer easily
express constraints of problem in terms of existing data structures and objects
on the heap. Despite having overhead of encoding and decoding, it is surprising
how is competitive Squander's SAT-based solution with specialized heuristic
developed for concrete problem.
%*****************************************************************************
\chapter{Logical programming principles}

The most known logical programming in the world is definitely Prolog and
thereby we can describe properties of logical programming using this language.
His name is derived from term PROgramming in LOGic and was developed for
programming of symbolic computation. His success led to formation new discipline
in mathematical information technology - \textbf{logical programming}.

Logical programming focus on description of relation's properties without need
to know how to do that.

\section{Logical programming paradigm}

Logical programming differ from imperative languages in following points:
\begin{enumerate}
  \item no assignment statement
  \item no cycles, no branching 
  \item no flow control
  \item object is marked as \textit{variable}, which satisfies some set of
  conditions, that are being during computation more specified
\end{enumerate}
Logical programming is based on following concepts:
\begin{enumerate}
  \item declaring facts about objects and relations between objects
  \item declaring valid rules about objects and relations between thmeselves and
  computing queries
\end{enumerate}
In logical programming are \textit{facts} unconditional commands and
\textit{rules} are conditional commands. Facts and rules are stored in one
shared database. Language does not differ between program and data.

\newpage
\subsection{Facts}
\label{sec:facts}
For expressing \textit{facts} and \textit{rules} are used \textit{clauses}.
Facts are used for expressing unconditionally true assertions and are clauses
with defined headers, but with no body. Usual way to ilustrate how to
composed facts are family relationships.

\begin{quote}
\begin{verbatim}
                            parent(john, jane).
                            parent(ann, jane).
                            parent(john, richard).
                            parent(ann, richard).
                            man(john).
                            man(richard).
                            womam(ann).
                            woman(jane).
\end{verbatim}
\end{quote}
Every clause declares concrete fact about relation. We can see that relation
\verb|parent|, e.g. \verb|parent(john, jane).| is concrete \textit{instance} of
this relation for \textit{objects} \verb|john| and \verb|jane|. After declaring
those facts is possible to form queries concerning relation \verb|parent|.

\begin{verbatim}
                                 ?-parent(john, jane).
\end{verbatim}
Example above shows, how it is possible to ask, if John is parent to Jane.
Because language has this fact recorded in its environment and answer is:

\begin{verbatim}                            
                                 yes
\end{verbatim}
Similarly can be query constructed on non-existing fact:

\begin{verbatim}
                                 ?-parent(john, emily).
                            
                                 no
\end{verbatim}
Answer to this query is \verb|no| of course. Query can be also composed in a
way, that we want to get some object, which is with other object in
required relation.

\begin{verbatim}
                                 ?-parent(X, jane).
                            
                                 X = john
\end{verbatim}
Here it is also possible to get other possible solution, so we get one more
positive answer:

\begin{verbatim}
                                 X = ann
\end{verbatim}
After that are all possible answers exhausted, so for the next command we get:

\begin{verbatim}
                                 no
\end{verbatim}
Now we try to express little bit complicated query: who is mother of Jane. This
query is necessary construct from two suqueries. First we limit set of solution
to Jane's parents. For that purpose we use query already shown above:

\begin{verbatim}
                                 ?-parent(X, jane).
\end{verbatim}


\newpage

In variable \verb|X| is now stored every object, who has relation \verb|parent|
to object \verb|jane|. In our case \verb|john| and \verb|ann|. Now we have to
limit this set of results to object, which is declared in clause \verb|women|,
so we add:

\begin{verbatim}
                          ?-parent(X, jane), women(X).
\end{verbatim}
As a result of these clauses we get:

\begin{verbatim}
                          X = ann
\end{verbatim}
and nothing more. In framework Squander are as facts used objects, that are
declared in framework's rules (see subsection \ref{sec:rules}). There are no
facts declared explicitly so we do not devote to them any more.

\subsection{Rules}
\label{sec:rules}
Expressing knowledge by facts cannot be always effective. Complexity of
relationships in family expressed by unconditional commands would lead to
big expansion of database. Despite having unlimited memory available, searching
for relevant information would has been time consuming. For this reason Prolog
provides conditional expressions - \textit{rules}.

By investigating facts is possible to derived new rule based on logical or
factual context. That knowledge allow us to express facts, which are not
explicitly stored in database. Let us show it in following example:

\begin{verbatim}
                          mother(X, Y) :- parent(X, Y),woman(X).
\end{verbatim}
Left side of rule expressed so called \textbf{head of rule} and right side
\textbf{body of rule}. This rule express, that \verb|X| is mother of \verb|Y|.
Rule is only labeled generalization of last example in previous subsection \ref{sec:facts}.
In next example will be shown more complicated construct:

\begin{verbatim}
               brother(X, Y) :- parent(O, Y),parent(O, X),man(X).
\end{verbatim}
Expression above means, that \verb|X| is brother of \verb|Y| if exists at least
one object \verb|O|, which is common in relation \verb|parent| for both \verb|X|
and \verb|Y| assuming \verb|X| is a man. Mathematical interpretation of teh
rule is:
\begin{center}
\uv{For all arbitrary persons X and Y,\\
 if some of the parents of X is O \\
 and some of the parents of Y is O \\
 and person X is a man, \\
 then X is brother of Y.}
\end{center}


When calling this rule with following parameters:
\begin{verbatim}
                          brother(richard, Y).
                          
                          Y = jane
\end{verbatim}
Rules are main construct of framework Squander, because they defined state of
object before and after computation, declares which object or object's
properties can be modified etc. On the other hand these rules are not called as
it is in Prolog, but they are declared as metadata for handling with objects.
More about Squander's rules in chapter \ref{chap:framework}.

\section{Terminology}
When simplyfying in Prolog, we can say, that in every task appear
\textit{objects} and \textit{relations}. Name of objects are called
\textit{terms} and name of relations are called \textit{predicates}. Terms are
analogous to arithmetic expressions, which point to the computed value, and
predicates are analogous to name of procedures, which defines relationship
between input parameters and output values, in imperative programming language.

There are two types of terms: \textit{simple terms} consisting of constants
(e.g. \verb|ann|, \verb|richard|,\ldots) and \textit{compound terms}. Compound
terms are also called \textit{structures} is every term containing simple
or compound term.

In previous subsection \ref{sec:rules}, there are predicates \verb|parent|,
\verb|man| and \verb|woman| as names of three relations defined by program.
Predicate with name \verb|parent| is defined as a set:
\verb|{(john,jane),(ann,jane),(john,richard),(ann,richard)}|. Next predicate
with name \verb|man| is defined as a set:
\verb|{john,richard}| and finally predicate
with name \verb|woman| is defined as a set:
\verb|{ann,jane}|. 

Finally there are three types of \textit{formulas} in Prolog:
\begin{itemize}
  \item \textit{atomic} - basic formulas (e.g. \verb|parent(john, jane).,|
  \verb|parent(ann, jane).|)
  \item \textit{conditional command} - implication constructs $A :- P_1,
  P_2,\ldots, P_n$ where $P_1, P_2,\ldots, P_n$ are atomic formulas
  (e.g. \verb|brother(X, Y) :- parent(O, Y),parent(O, X),man(X).|)
  \item \textit{target clauses} - query type ($?-C_1, C_2,\ldots, C_n$ where
  $C_1, C_2,\ldots, C_n$ are targets)\\ (e.g. \verb|?-parent(X, jane),|
  \verb|women(X).|)
\end{itemize} 
%*****************************************************************************
\chapter{Meaning of annotations}
Analýza a návrh implementace (včetně diskuse různých alternativ a volby implementačního prostředí).


%*****************************************************************************
\chapter{Framework description}
\label{chap:framework}
Popis implementace/realizace se zaměřením na nestandardní části řešení.

%*****************************************************************************
\chapter{Implementation of algorithms}
Popis implementace/realizace se zaměřením na nestandardní části řešení.


%*****************************************************************************
\chapter{Comparison}

\begin{itemize}
 \item Způsob, průběh a výsledky testování.
 \item Srovnání s existujícími řešeními, pokud jsou známy.
\end{itemize} 


%*****************************************************************************
\chapter{Conclusion}

\begin{itemize}
\item Zhodnocení splnění cílů DP/BP a  vlastního přínosu práce (při formulaci je třeba vzít v potaz zadání práce).
\item Diskuse dalšího možného pokračování práce.
\end{itemize} 

%*****************************************************************************
% Seznam literatury je v samostatnem souboru reference.bib. Ten
% upravte dle vlastnich potreb, potom zpracujte (a do textu
% zapracujte) pomoci prikazu bibtex a nasledne pdflatex (nebo
% latex). Druhy z nich alespon 2x, aby se poresily odkazy.

% originally following specification for bibliography formating was used
%\bibliographystyle{abbrv}

% Here is an improvment by Petr Dlouhy (April 2010).
% It is mainly for supervisors who expect Czech fomrating rules for references
% Additional feature is live url addresses to sources from your pdf file
% It requires the file csplainnat.bst (included in this sample zipfile).

\bibliographystyle{csplainnat}

%bibliographystyle{plain}
%\bibliographystyle{psc}
{
%JZ: 11.12.2008 Kdo chce mit v techto ukazkovych odkazech take odkaz na CSTeX:
\def\CS{$\cal C\kern-0.1667em\lower.5ex\hbox{$\cal S$}\kern-0.075em $}
\bibliography{reference}
}

% M. Dušek radi:
%\bibliographystyle{alpha}
% kdy citace ma tvar [AutorRok] (napriklad [Cook97]). Sice to asi neni  podle ceske normy (BTW BibTeX stejne neodpovida ceske norme), ale je to nejprehlednejsi.
% 3.5.2009 JZ polemizuje: BibTeX neobvinujte, napiste a poskytnete nam styl (.bst) splnujici citacni normu CSN/ISO.

%*****************************************************************************
%*****************************************************************************
\appendix

\chapter{Testování zaplnění stránky a odsazení odstavců}
\textbf{\large Tato příloha nebude součástí vaší práce. 
Slouží pouze jako příklad formátování textu.}

\section*{}
Určitě existuje nějaká pěkná latinská věta, která se k tomuhle testování používá, ale co mají dělat ti, kteří se nikdy latinsky neučili? Určitě existuje nějaká pěkná latinská věta, která se k tomuhle testování používá, ale co mají dělat ti, kteří se nikdy latinsky neučili? Určitě existuje nějaká pěkná latinská věta, která se k tomuhle testování používá, ale co mají dělat ti, kteří se nikdy latinsky neučili?

Určitě existuje nějaká pěkná latinská věta, která se k tomuhle testování používá, ale co mají dělat ti, kteří se nikdy latinsky neučili? Určitě existuje nějaká pěkná latinská věta, která se k tomuhle testování používá, ale co mají dělat ti, kteří se nikdy latinsky neučili? Určitě existuje nějaká pěkná latinská věta, která se k tomuhle testování používá, ale co mají dělat ti, kteří se nikdy latinsky neučili?

Určitě existuje nějaká pěkná latinská věta, která se k tomuhle testování používá, ale co mají dělat ti, kteří se nikdy latinsky neučili? Určitě existuje nějaká pěkná latinská věta, která se k tomuhle testování používá, ale co mají dělat ti, kteří se nikdy latinsky neučili? Určitě existuje nějaká pěkná latinská věta, která se k tomuhle testování používá, ale co mají dělat ti, kteří se nikdy latinsky neučili?

Určitě existuje nějaká pěkná latinská věta, která se k tomuhle testování používá, ale co mají dělat ti, kteří se nikdy latinsky neučili? Určitě existuje nějaká pěkná latinská věta, která se k tomuhle testování používá, ale co mají dělat ti, kteří se nikdy latinsky neučili? Určitě existuje nějaká pěkná latinská věta, která se k tomuhle testování používá, ale co mají dělat ti, kteří se nikdy latinsky neučili? Určitě existuje nějaká pěkná latinská věta, která se k tomuhle testování používá, ale co mají dělat ti, kteří se nikdy latinsky neučili? Určitě existuje nějaká pěkná latinská věta, která se k tomuhle testování používá, ale co mají dělat ti, kteří se nikdy latinsky neučili? Určitě existuje nějaká pěkná latinská věta, která se k tomuhle testování používá, ale co mají dělat ti, kteří se nikdy latinsky neučili?

Určitě existuje nějaká pěkná latinská věta, která se k tomuhle testování používá, ale co mají dělat ti, kteří se nikdy latinsky neučili? Určitě existuje nějaká pěkná latinská věta, která se k tomuhle testování používá, ale co mají dělat ti, kteří se nikdy latinsky neučili?

Určitě existuje nějaká pěkná latinská věta, která se k tomuhle testování používá, ale co mají dělat ti, kteří se nikdy latinsky neučili? Určitě existuje nějaká pěkná latinská věta, která se k tomuhle testování používá, ale co mají dělat ti, kteří se nikdy latinsky neučili? Určitě existuje nějaká pěkná latinská věta, která se k tomuhle testování používá, ale co mají dělat ti, kteří se nikdy latinsky neučili? Určitě existuje nějaká pěkná latinská věta, která se k tomuhle testování používá, ale co mají dělat ti, kteří se nikdy latinsky neučili? Určitě existuje nějaká pěkná latinská věta, která se k tomuhle testování používá, ale co mají dělat ti, kteří se nikdy latinsky neučili?

Určitě existuje nějaká pěkná latinská věta, která se k tomuhle testování používá, ale co mají dělat ti, kteří se nikdy latinsky neučili? Určitě existuje nějaká pěkná latinská věta, která se k tomuhle testování používá, ale co mají dělat ti, kteří se nikdy latinsky neučili? Určitě existuje nějaká pěkná latinská věta, která se k tomuhle testování používá, ale co mají dělat ti, kteří se nikdy latinsky neučili? Určitě existuje nějaká pěkná latinská věta, která se k tomuhle testování používá, ale co mají dělat ti, kteří se nikdy latinsky neučili? Určitě existuje nějaká pěkná latinská věta, která se k tomuhle testování používá, ale co mají dělat ti, kteří se nikdy latinsky neučili?

Určitě existuje nějaká pěkná latinská věta, která se k tomuhle testování používá, ale co mají dělat ti, kteří se nikdy latinsky neučili? Určitě existuje nějaká pěkná latinská věta, která se k tomuhle testování používá, ale co mají dělat ti, kteří se nikdy latinsky neučili? Určitě existuje nějaká pěkná latinská věta, která se k tomuhle testování používá, ale co mají dělat ti, kteří se nikdy latinsky neučili? Určitě existuje nějaká pěkná latinská věta, která se k tomuhle testování používá, ale co mají dělat ti, kteří se nikdy latinsky neučili? Určitě existuje nějaká pěkná latinská věta, která se k tomuhle testování používá, ale co mají dělat ti, kteří se nikdy latinsky neučili?

Určitě existuje nějaká pěkná latinská věta, která se k tomuhle testování používá, ale co mají dělat ti, kteří se nikdy latinsky neučili? Určitě existuje nějaká pěkná latinská věta, která se k tomuhle testování používá, ale co mají dělat ti, kteří se nikdy latinsky neučili? Určitě existuje nějaká pěkná latinská věta, která se k tomuhle testování používá, ale co mají dělat ti, kteří se nikdy latinsky neučili? Určitě existuje nějaká pěkná latinská věta, která se k tomuhle testování používá, ale co mají dělat ti, kteří se nikdy latinsky neučili? Určitě existuje nějaká pěkná latinská věta, která se k tomuhle testování používá, ale co mají dělat ti, kteří se nikdy latinsky neučili?

Určitě existuje nějaká pěkná latinská věta, která se k tomuhle testování používá, ale co mají dělat ti, kteří se nikdy latinsky neučili? Určitě existuje nějaká pěkná latinská věta, která se k tomuhle testování používá, ale co mají dělat ti, kteří se nikdy latinsky neučili? Určitě existuje nějaká pěkná latinská věta, která se k tomuhle testování používá, ale co mají dělat ti, kteří se nikdy latinsky neučili? Určitě existuje nějaká pěkná latinská věta, která se k tomuhle testování používá, ale co mají dělat ti, kteří se nikdy latinsky neučili? Určitě existuje nějaká pěkná latinská věta, která se k tomuhle testování používá, ale co mají dělat ti, kteří se nikdy latinsky neučili?

Určitě existuje nějaká pěkná latinská věta, která se k tomuhle testování používá, ale co mají dělat ti, kteří se nikdy latinsky neučili? Určitě existuje nějaká pěkná latinská věta, která se k tomuhle testování používá, ale co mají dělat ti, kteří se nikdy latinsky neučili? Určitě existuje nějaká pěkná latinská věta, která se k tomuhle testování používá, ale co mají dělat ti, kteří se nikdy latinsky neučili? Určitě existuje nějaká pěkná latinská věta, která se k tomuhle testování používá, ale co mají dělat ti, kteří se nikdy latinsky neučili? Určitě existuje nějaká pěkná latinská věta, která se k tomuhle testování používá, ale co mají dělat ti, kteří se nikdy latinsky neučili?

Určitě existuje nějaká pěkná latinská věta, která se k tomuhle testování používá, ale co mají dělat ti, kteří se nikdy latinsky neučili? Určitě existuje nějaká pěkná latinská věta, která se k tomuhle testování používá, ale co mají dělat ti, kteří se nikdy latinsky neučili? Určitě existuje nějaká pěkná latinská věta, která se k tomuhle testování používá, ale co mají dělat ti, kteří se nikdy latinsky neučili? Určitě existuje nějaká pěkná latinská věta, která se k tomuhle testování používá, ale co mají dělat ti, kteří se nikdy latinsky neučili? Určitě existuje nějaká pěkná latinská věta, která se k tomuhle testování používá, ale co mají dělat ti, kteří se nikdy latinsky neučili?

Určitě existuje nějaká pěkná latinská věta, která se k tomuhle testování používá, ale co mají dělat ti, kteří se nikdy latinsky neučili? Určitě existuje nějaká pěkná latinská věta, která se k tomuhle testování používá, ale co mají dělat ti, kteří se nikdy latinsky neučili? Určitě existuje nějaká pěkná latinská věta, která se k tomuhle testování používá, ale co mají dělat ti, kteří se nikdy latinsky neučili? Určitě existuje nějaká pěkná latinská věta, která se k tomuhle testování používá, ale co mají dělat ti, kteří se nikdy latinsky neučili? Určitě existuje nějaká pěkná latinská věta, která se k tomuhle testování používá, ale co mají dělat ti, kteří se nikdy latinsky neučili?

Určitě existuje nějaká pěkná latinská věta, která se k tomuhle testování používá, ale co mají dělat ti, kteří se nikdy latinsky neučili? Určitě existuje nějaká pěkná latinská věta, která se k tomuhle testování používá, ale co mají dělat ti, kteří se nikdy latinsky neučili? Určitě existuje nějaká pěkná latinská věta, která se k tomuhle testování používá, ale co mají dělat ti, kteří se nikdy latinsky neučili? Určitě existuje nějaká pěkná latinská věta, která se k tomuhle testování používá, ale co mají dělat ti, kteří se nikdy latinsky neučili? Určitě existuje nějaká pěkná latinská věta, která se k tomuhle testování používá, ale co mají dělat ti, kteří se nikdy latinsky neučili?

Určitě existuje nějaká pěkná latinská věta, která se k tomuhle testování používá, ale co mají dělat ti, kteří se nikdy latinsky neučili? Určitě existuje nějaká pěkná latinská věta, která se k tomuhle testování používá, ale co mají dělat ti, kteří se nikdy latinsky neučili? Určitě existuje nějaká pěkná latinská věta, která se k tomuhle testování používá, ale co mají dělat ti, kteří se nikdy latinsky neučili? Určitě existuje nějaká pěkná latinská věta, která se k tomuhle testování používá, ale co mají dělat ti, kteří se nikdy latinsky neučili? Určitě existuje nějaká pěkná latinská věta, která se k tomuhle testování používá, ale co mají dělat ti, kteří se nikdy latinsky neučili?

Určitě existuje nějaká pěkná latinská věta, která se k tomuhle testování používá, ale co mají dělat ti, kteří se nikdy latinsky neučili? Určitě existuje nějaká pěkná latinská věta, která se k tomuhle testování používá, ale co mají dělat ti, kteří se nikdy latinsky neučili? Určitě existuje nějaká pěkná latinská věta, která se k tomuhle testování používá, ale co mají dělat ti, kteří se nikdy latinsky neučili? Určitě existuje nějaká pěkná latinská věta, která se k tomuhle testování používá, ale co mají dělat ti, kteří se nikdy latinsky neučili? Určitě existuje nějaká pěkná latinská věta, která se k tomuhle testování používá, ale co mají dělat ti, kteří se nikdy latinsky neučili?

Určitě existuje nějaká pěkná latinská věta, která se k tomuhle testování používá, ale co mají dělat ti, kteří se nikdy latinsky neučili? Určitě existuje nějaká pěkná latinská věta, která se k tomuhle testování používá, ale co mají dělat ti, kteří se nikdy latinsky neučili? Určitě existuje nějaká pěkná latinská věta, která se k tomuhle testování používá, ale co mají dělat ti, kteří se nikdy latinsky neučili? Určitě existuje nějaká pěkná latinská věta, která se k tomuhle testování používá, ale co mají dělat ti, kteří se nikdy latinsky neučili? Určitě existuje nějaká pěkná latinská věta, která se k tomuhle testování používá, ale co mají dělat ti, kteří se nikdy latinsky neučili?

Určitě existuje nějaká pěkná latinská věta, která se k tomuhle testování používá, ale co mají dělat ti, kteří se nikdy latinsky neučili? Určitě existuje nějaká pěkná latinská věta, která se k tomuhle testování používá, ale co mají dělat ti, kteří se nikdy latinsky neučili? Určitě existuje nějaká pěkná latinská věta, která se k tomuhle testování používá, ale co mají dělat ti, kteří se nikdy latinsky neučili? Určitě existuje nějaká pěkná latinská věta, která se k tomuhle testování používá, ale co mají dělat ti, kteří se nikdy latinsky neučili? Určitě existuje nějaká pěkná latinská věta, která se k tomuhle testování používá, ale co mají dělat ti, kteří se nikdy latinsky neučili?

Určitě existuje nějaká pěkná latinská věta, která se k tomuhle testování používá, ale co mají dělat ti, kteří se nikdy latinsky neučili? Určitě existuje nějaká pěkná latinská věta, která se k tomuhle testování používá, ale co mají dělat ti, kteří se nikdy latinsky neučili? Určitě existuje nějaká pěkná latinská věta, která se k tomuhle testování používá, ale co mají dělat ti, kteří se nikdy latinsky neučili? Určitě existuje nějaká pěkná latinská věta, která se k tomuhle testování používá, ale co mají dělat ti, kteří se nikdy latinsky neučili? Určitě existuje nějaká pěkná latinská věta, která se k tomuhle testování používá, ale co mají dělat ti, kteří se nikdy latinsky neučili?

Určitě existuje nějaká pěkná latinská věta, která se k tomuhle testování používá, ale co mají dělat ti, kteří se nikdy latinsky neučili? Určitě existuje nějaká pěkná latinská věta, která se k tomuhle testování používá, ale co mají dělat ti, kteří se nikdy latinsky neučili? Určitě existuje nějaká pěkná latinská věta, která se k tomuhle testování používá, ale co mají dělat ti, kteří se nikdy latinsky neučili? Určitě existuje nějaká pěkná latinská věta, která se k tomuhle testování používá, ale co mají dělat ti, kteří se nikdy latinsky neučili? Určitě existuje nějaká pěkná latinská věta, která se k tomuhle testování používá, ale co mají dělat ti, kteří se nikdy latinsky neučili?

%*****************************************************************************
\chapter{Pokyny a návody k formátování textu práce}
\textbf{\large Tato příloha samozřejmě nebude součástí vaší práce. Slouží pouze jako příklad formátování textu.}

Používat se dají všechny příkazy systému \LaTeX. Existuje velké množství volně přístupné dokumentace, tutoriálů, příruček a dalších materiálů v elektronické podobě. Výchozím bodem, kromě Googlu, může být stránka CSTUG (Czech Tech Users Group) \cite{CSTUG}. Tam najdete odkazy na další materiály.  Vetšinou dostačující a přehledně organizovanou elektronikou dokumentaci najdete například na \cite{latexdocweb} nebo \cite{latexwiki}.

Existují i různé nadstavby nad systémy \TeX{} a \LaTeX, které výrazně usnadní psaní textu zejména začátečníkům. Velmi rozšířený v Linuxovém prostředí je systém Kile.


\section{Vkládání obrázků}
Obrázky se umísťují do plovoucího prostředí \verb|figure|. Každý obrázek by měl obsahovat \textbf{název} (\verb|\caption|) a \textbf{návěští} (\verb|\label|). Použití příkazu pro vložení obrázku \\\verb|\includegraphics| je podmíněno aktivací (načtením) balíku graphicx příkazem\\ \verb|\usepackage{graphicx}|.

Budete-li zdrojový text zpracovávat pomocí programu \verb|pdflatex|, očekávají se obrázky s příponou \verb|*.pdf|\footnote{pdflatex umí také formáty PNG a JPG.}, použijete-li k formátování \verb|latex|, očekávají se obrázky s příponou \verb|*.eps|.\footnote{Vzájemnou konverzi mezi snad všemi typy obrazku včetně změn vekostí a dalších vymožeností vám může zajistit balík ImageMagic  (http://www.imagemagick.org/script/index.php). Je dostupný pod Linuxem, Mac OS i MS Windows. Důležité jsou zejména příkazy convert a identify.}

\begin{figure}[ht]
\begin{center}
\includegraphics[width=5cm]{figures/LogoCVUT}
\caption{Popiska obrázku}
\label{fig:logo}
\end{center}
\end{figure}

Příklad vložení obrázku:
\begin{verbatim}
\begin{figure}[h]
\begin{center}
\includegraphics[width=5cm]{figures/LogoCVUT}
\caption{Popiska obrazku}
\label{fig:logo}
\end{center}
\end{figure}
\end{verbatim}

\section{Kreslení obrázků}
Zřejmě každý z vás má nějaký oblíbený nástroj pro tvorbu obrázků. Jde jen o to, abyste dokázali obrázek uložit v požadovaném formátu nebo jej do něj konvertovat (viz předchozí kapitola). Je zřejmě vhodné kreslit obrázky vektorově. Celkem oblíbený, na ovládání celkem jednoduchý a přitom dostatečně mocný je například program Inkscape.

Zde stojí za to upozornit na kreslící programe Ipe \cite{ipe}, který dokáže do obrázku vkládat komentáře přímo v latexovském formátu (vzroce, stejné fonty atd.). Podobné věci umí na Linuxové platformě nástroj Xfig. 

Za pozornost ještě stojí schopnost editoru Ipe importovat obrázek (jpg nebo bitmap) a krelit do něj latexovské popisky a komentáře. Výsledek pak umí exportovat přímo do pdf.

\section{Tabulky}
Existuje více způsobů, jak sázet tabulky. Například je možno použít prostředí \verb|table|, které je velmi podobné prostředí \verb|figure|. 

\begin{table}
\begin{center}
\begin{tabular}{|c|l|l|}
\hline
\textbf{DTD} & \textbf{construction} & \textbf{elimination} \\
\hline
$\mid$ & \verb+in1|A|B a:sum A B+ & \verb+case([_:A]a)([_:B]a)ab:A+\\
&\verb+in1|A|B b:sum A B+ & \verb+case([_:A]b)([_:B]b)ba:B+\\
\hline
$+$&\verb+do_reg:A -> reg A+&\verb+undo_reg:reg A -> A+\\
\hline
$*,?$& the same like $\mid$ and $+$ & the same like $\mid$ and $+$\\
& with \verb+emtpy_el:empty+ & with \verb+emtpy_el:empty+\\
\hline
R(a,b) & \verb+make_R:A->B->R+ & \verb+a: R -> A+\\
 & & \verb+b: R -> B+\\
\hline
\end{tabular}
\end{center}
\caption{Ukázka tabulky}
\label{tab:tab1}
\end{table}

Zdrojový text tabulky \ref{tab:tab1} vypadá takto:
\begin{verbatim}
\begin{table}
\begin{center}
\begin{tabular}{|c|l|l|}
\hline
\textbf{DTD} & \textbf{construction} & \textbf{elimination} \\
\hline
$\mid$ & \verb+in1|A|B a:sum A B+ & \verb+case([_:A]a)([_:B]a)ab:A+\\
&\verb+in1|A|B b:sum A B+ & \verb+case([_:A]b)([_:B]b)ba:B+\\
\hline
$+$&\verb+do_reg:A -> reg A+&\verb+undo_reg:reg A -> A+\\
\hline
$*,?$& the same like $\mid$ and $+$ & the same like $\mid$ and $+$\\
& with \verb+emtpy_el:empty+ & with \verb+emtpy_el:empty+\\
\hline
R(a,b) & \verb+make_R:A->B->R+ & \verb+a: R -> A+\\
 & & \verb+b: R -> B+\\
\hline
\end{tabular}
\end{center}
\caption{Ukázka tabulky}
\label{tab:tab1}
\end{table}
\begin{table}
\end{verbatim}

\section{Odkazy v textu}
\subsection{Odkazy na literaturu}
Jsou realizovány příkazem \verb|\cite{odkaz}|. 

Seznam literatury je dobré zapsat do samostatného souboru a ten pak zpracovat programem bibtex (viz soubor \verb|reference.bib|). Zdrojový soubor pro \verb|bibtex| vypadá například takto:
\begin{verbatim}
@Article{Chen01,
  author  = "Yong-Sheng Chen and Yi-Ping Hung and Chiou-Shann Fuh",
  title   = "Fast Block Matching Algorithm Based on 
             the Winner-Update Strategy",
  journal = "IEEE Transactions On Image Processing",
  pages   = "1212--1222",
  volume  =  10,
  number  =   8,
  year    = 2001,
}

@Misc{latexdocweb,
  author  = "",
  title   = "{\LaTeX} --- online manuál",
  note    = "\verb|http://www.cstug.cz/latex/lm/frames.html|",
  year    = "",
}
...
\end{verbatim}

%11.12.2008, 3.5.2009
\textbf{Pozor:} Sazba názvů odkazů je dána Bib\TeX{} stylem\\ (\verb|\bibliographystyle{abbrv}|). 
%Budete-li používat české prostředí (\verb|\usepackage[czech]{babel}|), 
Bib\TeX{} tedy obvykle vysází velké pouze počáteční písmeno z názvu zdroje, 
ostatní písmena zůstanou malá bez ohledu na to, jak je napíšete. 
Přesněji řečeno, styl může zvolit pro každý typ publikace jiné konverze. 
Pro časopisecké články třeba výše uvedené, jiné pro monografie (u nich často bývá 
naopak velikost písmen zachována).

Pokud chcete Bib\TeX u napovědět, která písmena nechat bez konverzí 
(viz \texttt{title = "\{$\backslash$LaTeX\} -{}-{}- online manuál"} 
v~předchozím příkladu), je nutné příslušné písmeno (zde celé makro) uzavřít 
do složených závorek. Pro přehlednost je proto vhodné celé parametry 
uzavírat do uvozovek (\texttt{author = "\dots"}), nikoliv do složených závorek.

Odkazy na literaturu ve zdrojovém textu se pak zapisují:
\begin{verbatim}
Podívejte se na \cite{Chen01}, 
další detaily najdete na \cite{latexdocweb}
\end{verbatim}

Vazbu mezi soubory \verb|*.tex| a \verb|*.bib| zajistíte příkazem 
\verb|\bibliography{}| v souboru \verb|*.tex|.  V našem případě tedy zdrojový 
dokument \verb|thesis.tex| obsahuje příkaz\\
\verb|\bibliography{reference}|.

Zpracování zdrojového textu s odkazy se provede postupným voláním programů\\
\verb|pdflatex <soubor>| (případně \verb|latex <soubor>|), \verb|bibtex <soubor>| 
a opět\\ \verb|pdflatex <soubor>|.\footnote{První volání \texttt{pdflatex} 
vytvoří soubor s~koncovkou \texttt{*.aux}, který je vstupem pro program 
\texttt{bibtex}, pak je potřeba znovu zavolat program \texttt{pdflatex} 
(\texttt{latex}), který tentokrát zpracuje soubory s příponami \texttt{.aux} a 
\texttt{.tex}. 
Informaci o případných nevyřešených odkazech (cross-reference) vidíte přímo při 
zpracovávání zdrojového souboru příkazem \texttt{pdflatex}. Program \texttt{pdflatex} 
(\texttt{latex}) lze volat vícekrát, pokud stále vidíte nevyřešené závislosti.}


Níže uvedený příklad je převzat z dříve existujících pokynů studentům, kteří 
dělají svou diplomovou nebo bakalářskou práci v~Grafické skupině.\footnote{Několikrát 
jsem byl upozorněn, že web s těmito pokyny byl zrušen, proto jej zde přímo necituji. 
Nicméně příklad sám o sobě dokumentuje obecně přijímaný konsensus ohledně citací 
v~bakalářských a diplomových pracích na KP.} Zde se praví:
\begin{small}
\begin{verbatim}
...
j) Seznam literatury a dalších použitých pramenů, odkazy na WWW stránky, ...
 Pozor na to, že na veškeré uvedené prameny se musíte v textu práce 
 odkazovat -- [1]. 
Pramen, na který neodkazujete, vypadá, že jste ho vlastně nepotřebovali 
a je uveden jen do počtu. Příklad citace knihy [1], článku v časopise [2], 
stati ve sborníku [3] a html odkazu [4]: 
[1] J. Žára, B. Beneš;, and P. Felkel. 
     Moderní počítačová grafika. Computer Press s.r.o, Brno, 1 edition, 1998. 
     (in Czech). 
[2] P. Slavík. Grammars and Rewriting Systems as Models for Graphical User 
     Interfaces. Cognitive Systems, 4(4--3):381--399, 1997. 
[3] M. Haindl, Š. Kment, and P. Slavík. Virtual Information Systems. 
     In WSCG'2000 -- Short communication papers, pages 22--27, Pilsen, 2000. 
     University of West Bohemia. 
[4] Knihovna grafické skupiny katedry počítačů: 
     http://www.cgg.cvut.cz/Bib/library/ 
\end{verbatim}
\end{small}
\ldots{} abychom výše citované odkazy skutečně našli v (automaticky generovaném) seznamu literatury tohoto textu, musíme je nyní alespoň jednou citovat: Kniha \cite{kniha}, článek v~časopisu \cite{clanek}, příspěvek na konferenci \cite{sbornik}, www odkaz \cite{www}.

Ještě přidáme další ukázku citací online zdrojů podle české normy. Odkaz na wiki o frameworcich \cite{wiki:framework} a ORM \cite{wiki:orm}. Použití viz soubor \verb|reference.bib|. V seznamu literatury by nyní měly být živé odkazy na zdroje. V \verb|reference.bib| je zcela nový typ publikace. Detaily dohledal a dodal Petr Dlouhý v dubnu 2010. Podrobnosti najdete ve zdrojovém souboru tohoto textu v komentáři u příkazu \verb|\thebibliography|.

\subsection{Odkazy na obrázky, tabulky a kapitoly}
\begin{itemize}
\item Označení místa v textu, na které chcete později čtenáře práce odkázat, se provede příkazem \verb|\label{navesti}|. Lze použít v prostředích \verb|figure| a  \verb|table|, ale též za názvem kapitoly nebo podkapitoly.
\item Na návěští se odkážeme příkazem \verb|\ref{navesti}| nebo \verb|\pageref{navesti}|.
\end{itemize}

\section{Rovnice, centrovaná, číslovaná matematika}
Jednoduchý matematický výraz zapsaný přímo do textu se vysází pomocí prostředí \verb|math|, resp. zkrácený zápis pomocí uzavření textu rovnice mezi znaky \verb|$|.

Kód \verb|$ S = \pi * r^2 $| bude vysázen takto: $ S = \pi * r^2 $.

Pokud chcete nečíslované rovnice, ale umístěné centrovaně na samostatné řádky, pak lze použít prostředí \verb|displaymath|, resp. zkrácený zápis pomocí uzavření textu rovnice mezi znaky \verb|$$|. Zdrojový kód: 
\begin{verb}
|$$ S = \pi * r^2 $$|
\end{verb}
bude pak vysázen takto:
$$ S = \pi * r^2 $$

Chcete-li mít rovnice číslované, je třeba použít prostředí \verb|eqation|. Kód:
\begin{verbatim}
\begin{equation}
  S = \pi * r^2
\end{equation}

\begin{equation}
  V = \pi * r^3
\end{equation}
\end{verbatim}
je potom vysázen takto:
\begin{equation}
  S = \pi * r^2
\end{equation}

\begin{equation}
  V = \pi * r^3
\end{equation}

\section{Kódy programu}
Chceme-li vysázet například část zdrojového kódu programu (bez formátování), hodí se prostředí \verb|verbatim|: 
\begin{verbatim}
         (* nickname2 *)
Lego> Refine in1
             (do_reg (nickname1 h));
Refine by  in1 (do_reg (nickname1 h))
   ?4 : pcdata
   ?5 : pcdata
          (* surname2 *)
Lego> Refine surname1 h;
Refine by  surname1 h
   ?5 : pcdata
          (* email2 *)
Lego> Refine undo_reg (email1 h);
Refine by  undo_reg (email1 h)
*** QED ***
\end{verbatim}

\section{Další poznámky}
\subsection{České uvozovky}
V souboru \verb|k336_thesis_macros.tex| je příkaz \verb|\uv{}| pro sázení českých uvozovek. \uv{Text uzavřený do českých uvozovek.}

% JZ: 3.5.2009 \chapter z book zajistí automaticky
%\subsection{Začátky kapitol na liché stránky}
%Ve výsledném textu je dobré, když každá kapitola začíná na liché stránce. Tedy použijte:
%\begin{verbatim}
%  \cleardoublepage\include{1_uvod}
%  \cleardoublepage\include{2_teorie}
%   atd.\ldots{}
%\end{verbatim}

%*****************************************************************************
\chapter{Seznam použitých zkratek}

\begin{description}
\item[2D] Two-Dimensional
\item[ABN] Abstract Boolean Networks
\item[ASIC] Application-Specific Integrated Circuit
\end{description}
\vdots

%*****************************************************************************
\chapter{UML diagramy}
\textbf{\large Tato příloha není povinná a zřejmě se neobjeví v každé práci. Máte-li ale větší množství podobných diagramů popisujících systém, není nutné všechny umísťovat do hlavního textu, zvláště pokud by to snižovalo jeho čitelnost.}

%*****************************************************************************
\chapter{Instalační a uživatelská příručka}
\textbf{\large Tato příloha velmi žádoucí zejména u softwarových implementačních prací.}

%*****************************************************************************
\chapter{Obsah přiloženého CD}
\textbf{\large Tato příloha je povinná pro každou práci. Každá práce musí totiž obsahovat přiložené CD. Viz dále.}

Může vypadat například takto. Váš seznam samozřejmě bude odpovídat typu vaší práce. (viz \cite{infodp}):

\begin{figure}[h]
\begin{center}
\includegraphics[width=14cm]{figures/seznamcd}
\caption{Seznam přiloženého CD --- příklad}
\label{fig:seznamcd}
\end{center}
\end{figure}

Na GNU/Linuxu si strukturu přiloženého CD můžete snadno vyrobit příkazem:\\ 
\verb|$ tree . >tree.txt|\\
Ve vzniklém souboru pak stačí pouze doplnit komentáře.

Z \textbf{README.TXT} (případne index.html apod.)  musí být rovněž zřejmé, jak programy instalovat, spouštět a jaké požadavky mají tyto programy na hardware.

Adresář \textbf{text}  musí obsahovat soubor s vlastním textem práce v PDF nebo PS formátu, který bude později použit pro prezentaci diplomové práce na WWW.

\end{document}
